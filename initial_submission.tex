\documentclass[10pt,twocolumn,letterpaper]{article}

\usepackage{3dv}
\usepackage{times}
\usepackage{epsfig}
\usepackage{graphicx}
\usepackage{amsmath}
\usepackage{amssymb}

% added packages
\usepackage{subcaption}
\usepackage{multirow}
\usepackage{array}
\usepackage{slashbox}
% Include other packages here, before hyperref.

% If you comment hyperref and then uncomment it, you should delete
% egpaper.aux before re-running latex.  (Or just hit 'q' on the first latex
% run, let it finish, and you should be clear).
\usepackage[pagebackref=true,breaklinks=true,letterpaper=true,colorlinks,bookmarks=false]{hyperref}


%\threedvfinalcopy % *** Uncomment this line for the final submission

\def\threedvPaperID{****} % *** Enter the 3DV Paper ID here
\def\httilde{\mbox{\tt\raisebox{-.5ex}{\symbol{126}}}}

% Pages are numbered in submission mode, and unnumbered in camera-ready
\ifthreedvfinal\pagestyle{empty}\fi
\begin{document}

%%%%%%%%% TITLE
\title{Recurrent Neural Network based Sign Language Recognition using 3D Skeleton Data}

\author{First Author\\
Institution1\\
Institution1 address\\
{\tt\small firstauthor@i1.org}
% For a paper whose authors are all at the same institution,
% omit the following lines up until the closing ``}''.
% Additional authors and addresses can be added with ``\and'',
% just like the second author.
% To save space, use either the email address or home page, not both
\and
Second Author\\
Institution2\\
First line of institution2 address\\
{\tt\small secondauthor@i2.org}
}

\maketitle
% \thispagestyle{empty}

%%%%%%%%% ABSTRACT
\begin{abstract}
Video based sign language recognition is a mature problem in the computer vision community. With the recent upsurge of depth sensor, $3D$ video data -- RGB with depth -- bas become popular choice for video analysis because of its extra depth modality. Another type of $3D$ data -- $3D$ skeleton data -- has shown extensive usage in human activity understanding and become part of state-of-art methods in this domain. Despite having similarity with the human activity recognition, use of $3D$ skeleton data in sign language recognition is scarce. One of the few reasons behind this is unavailability of benchmark public dataset. In this paper, we introduce a $3D$ skeleton dataset for American sign language (ASL) recognition. 
\end{abstract}

%%%%%%%%% BODY TEXT
\section{Introduction}

According to National Institute on Deafness and Other Communication Disorders, one in $1000$ infants is born totally deaf, while an additional one to six per thousand are born with hearing loss of different levels ~\cite{3072291}. Sign language is used by Deaf or Hard-of-Hearing (DHH) people to communicate among themselves. In addition to hearing difficulties, DHH people can have hardness in understanding and speaking natural languages spoken by normal hearing people ~\cite{doi:10.1080/01690965.2012.705006}. Because of that, they can not use spoken or written form of natural language to communicate with others. Sign language is the natural language to DHH people. Primary use of sign language is to communicate with other DHH people. It also allows them to develop intellectual abilities and learn social traits which are fundamental attributes any normal human being should posses as a part of society.

American sign language (ASL) is the $3^{rd}$ mostly used language in USA and is used by around half a million of people ~\cite{sign_lang_study}. Sign language can also be used by the DHH people to communicate with the normal hearing people given that both parties know it. But not many normal hearing people know sign language. That's why daily life communication with people around becomes challenging for them. For example, requesting someone around to open the door, turning on the thermostat or asking someone about weather condition etc. On the other hand, personal digital assistants (PDA) such as Siri, Cortana, Google Home etc. are becoming popular in making life easier for people. These PDAs can be used to make better quality of life for DHH people. To achieve this goal, we must need an ASL recognizer embedded with PDAs.

The purpose of this work is to build an ASL recognition system. Most of the current  system dealing with ASL recognition use RGB video data. ASL sign is performed by moving the hands combined with facial expressions and postures of the body. While video data is good for having a total view of whole body movement, we have noticed that, the motion of specific position of different locations of the hands and head areas are important for ASL recognition. In other word, ASL sign can be treated as a sequence of motion of some specific body parts. Using, video data it is not practical to single out each different body location and motion sequences associated with them from sequences of raw RGB value. Microsoft kinect is a $3D$ camera sensor which can use the $3D$ information of a person to provide specific $3D$ coordinate of a body location which is called skeletal data ~\cite{Zhang:2012:MKS:2225053.2225203}. To the best of our knowledge, there is no publicly available skeletal dataset in literature. We used the Kinect sensor to collect data for ASL recognition problem. 

With skeleton data, ASL can be seen as a sequence $3D$ coordinates or a $3D$ time series~\cite{7298714}. Accurate recognition of such data depends on how well we can model inherent sequential pattern in data. Recurrent neural network (RNN) has shown very good performance in type kind of sequential modeling~\cite{DBLP:journals/corr/Lipton15}. In this work, we are going to implement two different architectures which are variants of RNN with skeletal ASL data. Success of a practical machine learning system depends on it's performance after it is being deployed. In other words, the system must do well with new subjects other than those were used in training the system. To work in this direction, we proposed a calibration mechanism with our implemented system which essentially finds what fraction of data from a new subject is needed to give a performance boost to our system. Overall we have the following contributions in this work.
\begin{itemize}  
	\item We created a new dataset for ASL and made it public
	\item Implementation of different RNN variants to solve ASL recognition problem with skeletal data.
	\item Lastly, we propose a calibration mechanism which will tell us what percentage of data from a test subject is needed to give a performance boost to the system.
\end{itemize}

\section{Related work}
Most of the sign language recognition system in the literature use RGB video data as input. These methods use Hidden Markov Models (HMM) to model the sequential dependency for the recognition system. For example Zafrulla \etal used worn colored gloves in hands during data collection and developed an HMM based framework for ASL phrase verification ~\cite{copycat_zafrulla}. They also used hand crafted features from Kinect skeleton data and accelerometer worn in hand ~\cite{Zafrulla:2011:ASL:2070481.2070532}. Primarily they used directional unit vectors from one joint to another and angles between different combinations of three joints. Having all features crafted, they used an HMM based framework to build the system. Huang \etal showed effectiveness of using Convolutional neural network (CNN) with RGB video data for sign language recognition~\cite{7177428}. They used 3D CNN ~\cite{Ji:2013:CNN:2412386.2412939} to extract spatio-temporal features from video and their method does not depend on hand crafted feature for ASL recognition. Similar type of architecture was implemented by Pigou \etal for Italian gestures \cite{978-3-319-16178-5_40}. Sun \etal adopted a different way with RGB video data ~\cite{Sun:2015:LSV:2753829.2629481}. They hypothesized that, not all RGB frames in a video are equally important and assigned a binary latent variable to each frame in training videos for indicating representativeness of that frame and developed a latent support vector machine model for ASL recognition. Zaki \etal proposed two new features with existing hand crafted feature and developed the system using HMM based approach \cite{ZAKI2011572}. Cooper \etal used appearance based features and divided the whole system into sub units of RGB and tracking data \cite{Cooper2017}. They also used HMM based model to solve the ASL recognition problem. 
%All of the methods discussed so far mostly used RGB video data and HMM as their model. Although some authors tried to use tracking information, they used it along with RGB data. 

To our knowledge, no prior work has used data of tracking body locations alone for ASL recognition problem. However, in a closely similar task of human action recognition, a significant amount of work has been done using tracking body joint information. Sharoudy \etal developed the largest dataset in human activity recognition in \cite{7780484}. They also proposed an extension of long short term memory (LSTM) memory model which leverages group motion of several body joints to recognize human activity from joint tracking (skeletal) data. A different adaptation of the LSTM model was proposed by Jiu \etal ~\cite{8101019} where spatial interaction among joints was considered in addition to temporal dynamics to achieve higher recognition performance. Veeriah \etal proposed a LSTM network to capture salient motion pattern of body joints~\cite{7410817}. This method takes into account the derivative of motion states associated with different body joints. Du \etal treated the whole body as a hierarchical configuration of different body parts and proposed a hierarchical RNN to recognize human activities~\cite{7298714}. Some attention based model were also proposed for human activity analysis ~\cite{8226767, song2016end}. There are some methods which converted skeleton sequences of body joints or RGB videos into an image representation and then applied state-of-art image recognition models to achieve good performance \cite{DBLP:conf/cvpr/KeBASB17, DBLP:journals/corr/abs-1711-05941}.
%All of the works mentioned here either dealt with RGB video data in case of ASL recognition or used skeleton data for human activity recognition. 
Motivated by successful use of skeleton data in activity recognition, our approach is to perform ASL recognition using skeletal data collected from Kinect sensor.  

%-------------------------------------------------------------------------
%\subsection{Language}
%
%All manuscripts must be in English.
%
%\subsection{Dual submission}
%
%By submitting a manuscript to 3DV, the authors assert that it has not been
%previously published in substantially similar form. Furthermore, no paper
%which contains significant overlap with the contributions of this paper
%either has been or will be submitted during the 3DV 2018 review period to
%{\bf either a journal} or any conference or any
%workshop.  {\bf Papers violating this condition will be rejected}.
%
%If there are papers that may appear to the reviewers
%to violate this condition, then it is your responsibility to: (1)~cite
%these papers (preserving anonymity as described in Section 1.6 below),
%(2)~argue in the body of your paper why your 3DV paper is non-trivially
%different from these concurrent submissions, and (3)~include anonymized
%versions of those papers in the supplemental material.
%
%\subsection{Paper length}
%3DV papers should be no longer than 8 pages, excluding references.
%The references section will not be included in the page count, and
%there is no limit on the length of the references section. Overlength
%papers will simply not be reviewed.  This includes papers where the
%margins and formatting are deemed to have been significantly altered
%from those laid down by this style guide.  Note that this \LaTeX\ 
%guide already sets figure captions and references in a smaller font.
%The reason such papers will not be reviewed is that there is no
%provision for supervised revisions of manuscripts.  The reviewing
%process cannot determine the suitability of the paper for presentation
%in eight pages if it is reviewed in eleven.
%
%%-------------------------------------------------------------------------
%\subsection{The ruler}
%The \LaTeX\ style defines a printed ruler which should be present in the
%version submitted for review.  The ruler is provided in order that
%reviewers may comment on particular lines in the paper without
%circumlocution.  If you are preparing a document using a non-\LaTeX\
%document preparation system, please arrange for an equivalent ruler to
%appear on the final output pages.  The presence or absence of the ruler
%should not change the appearance of any other content on the page.  The
%camera ready copy should not contain a ruler. (\LaTeX\ users may uncomment
%the \verb'\threedvfinalcopy' command in the document preamble.)  Reviewers:
%note that the ruler measurements do not align well with lines in the paper
%--- this turns out to be very difficult to do well when the paper contains
%many figures and equations, and, when done, looks ugly.  Just use fractional
%references (e.g.\ this line is $097.5$), although in most cases one would
%expect that the approximate location will be adequate.
%
%\subsection{Mathematics}
%
%Please number all of your sections and displayed equations.  It is
%important for readers to be able to refer to any particular equation.  Just
%because you didn't refer to it in the text doesn't mean some future reader
%might not need to refer to it.  It is cumbersome to have to use
%circumlocutions like ``the equation second from the top of page 3 column
%1''.  (Note that the ruler will not be present in the final copy, so is not
%an alternative to equation numbers).  All authors will benefit from reading
%Mermin's description of how to write mathematics:
%\url{http://www.pamitc.org/documents/mermin.pdf}.  
%
%
%\subsection{Blind review}
%
%Many authors misunderstand the concept of anonymizing for blind
%review.  Blind review does not mean that one must remove
%citations to one's own work---in fact it is often impossible to
%review a paper unless the previous citations are known and
%available.
%
%Blind review means that you do not use the words ``my'' or ``our''
%when citing previous work.  That is all.  (But see below for
%techreports)
%
%Saying ``this builds on the work of Lucy Smith [1]'' does not say
%that you are Lucy Smith, it says that you are building on her
%work.  If you are Smith and Jones, do not say ``as we show in
%[7]'', say ``as Smith and Jones show in [7]'' and at the end of the
%paper, include reference 7 as you would any other cited work.
%
%An example of a bad paper just asking to be rejected:
%\begin{quote}
%\begin{center}
%    An analysis of the frobnicatable foo filter.
%\end{center}
%
%   In this paper we present a performance analysis of our
%   previous paper [1], and show it to be inferior to all
%   previously known methods.  Why the previous paper was
%   accepted without this analysis is beyond me.
%
%   [1] Removed for blind review
%\end{quote}
%
%
%An example of an acceptable paper:
%
%\begin{quote}
%\begin{center}
%     An analysis of the frobnicatable foo filter.
%\end{center}
%
%   In this paper we present a performance analysis of the
%   paper of Smith \etal [1], and show it to be inferior to
%   all previously known methods.  Why the previous paper
%   was accepted without this analysis is beyond me.
%
%   [1] Smith, L and Jones, C. ``The frobnicatable foo
%   filter, a fundamental contribution to human knowledge''.
%   Nature 381(12), 1-213.
%\end{quote}
%
%If you are making a submission to another conference at the same time,
%which covers similar or overlapping material, you may need to refer to that
%submission in order to explain the differences, just as you would if you
%had previously published related work.  In such cases, include the
%anonymized parallel submission~\cite{Authors12} as additional material and
%cite it as
%\begin{quote}
%[1] Authors. ``The frobnicatable foo filter'', F\&G 2018 Submission ID 324,
%Supplied as additional material {\tt fg324.pdf}.
%\end{quote}
%
%Finally, you may feel you need to tell the reader that more details can be
%found elsewhere, and refer them to a technical report.  For conference
%submissions, the paper must stand on its own, and not {\em require} the
%reviewer to go to a techreport for further details.  Thus, you may say in
%the body of the paper ``further details may be found
%in~\cite{Authors12b}''.  Then submit the techreport as additional material.
%Again, you may not assume the reviewers will read this material.
%
%Sometimes your paper is about a problem which you tested using a tool which
%is widely known to be restricted to a single institution.  For example,
%let's say it's 1969, you have solved a key problem on the Apollo lander,
%and you believe that the 3DV70 audience would like to hear about your
%solution.  The work is a development of your celebrated 1968 paper entitled
%``Zero-g frobnication: How being the only people in the world with access to
%the Apollo lander source code makes us a wow at parties'', by Zeus \etal.
%
%You can handle this paper like any other.  Don't write ``We show how to
%improve our previous work [Anonymous, 1968].  This time we tested the
%algorithm on a lunar lander [name of lander removed for blind review]''.
%That would be silly, and would immediately identify the authors. Instead
%write the following:
%\begin{quotation}
%\noindent
%   We describe a system for zero-g frobnication.  This
%   system is new because it handles the following cases:
%   A, B.  Previous systems [Zeus et al. 1968] didn't
%   handle case B properly.  Ours handles it by including
%   a foo term in the bar integral.
%
%   ...
%
%   The proposed system was integrated with the Apollo
%   lunar lander, and went all the way to the moon, don't
%   you know.  It displayed the following behaviours
%   which show how well we solved cases A and B: ...
%\end{quotation}
%As you can see, the above text follows standard scientific convention,
%reads better than the first version, and does not explicitly name you as
%the authors.  A reviewer might think it likely that the new paper was
%written by Zeus \etal, but cannot make any decision based on that guess.
%He or she would have to be sure that no other authors could have been
%contracted to solve problem B.
%
%FAQ: Are acknowledgements OK?  No.  Leave them for the final copy.


%\begin{figure}[t]
%\begin{center}
%\fbox{\rule{0pt}{2in} \rule{0.9\linewidth}{0pt}}
%   %\includegraphics[width=0.8\linewidth]{egfigure.eps}
%\end{center}
%   \caption{Example of caption.  It is set in Roman so that mathematics
%   (always set in Roman: $B \sin A = A \sin B$) may be included without an
%   ugly clash.}
%\label{fig:long}
%\label{fig:onecol}
%\end{figure}
\section{Modeling sign language}

%Most of the dataset in literature used RGB video in ASL recognition. 
Most of the current sign language recognition system use RGB video data as input. Using RGB data there is no way to track a specific body location, such as hand tip, wrist, neck etc. But, to correctly recognize gestures, motion trajectory of each body part could be useful. With the advent of Microsoft Kinect sensor, it is possible to track several specific body locations. Specifically, Kinect version $2$ can track $25$ body joints. Kinect does this by exploiting depth information in video and using a machine learning model embedded inside it. This process is called skeletal tracking. Figure~\ref{fig:kinect_sk} shows how Kinect sees a person as a configuration of skeletal joints \footnote{https://msdn.microsoft.com/en-us/library/microsoft.kinect.jointtype.aspx}. 
\begin{figure}[h]
	\begin{center}
		\includegraphics[width=.8\linewidth]{kinect_sk}
	\end{center}
	\caption{Kinect skeletal body configuration}
	\label{fig:kinect_sk}
\end{figure}
Since no skeletal dataset was publicly available for ASL, we built our own dataset to initiate research in this direction and made the dataset publicly available.  
\subsection{Data collection}
We collected $480$ ASL samples for $\number4$ signers and $\number6$ daily activities related signs. Since our main goal of this work is to help deaf and hearing impaired people, we chose six activities that are considered important for them in daily life. Those are listed in Table~\ref{table:asl_signs}. For each person, we collected $\number20$ samples for each sign. In this way, we ended up with total $20 \times 6 \times 4$ data samples. We used Kinect version $2$ sensor for data collection. For each performer we put the sensor in different distances from the person to make the dataset more challenging and realistic. Figure~\ref{fig:ac_person1} shows an example of data. It shows six frames from the gesture \textit{air conditioning} at the top and skeletal joints drawn to them at the bottom.

\begin{table}[h]
	\begin{center}
		\begin{tabular}{|c|c|c}
			\hline
			Wake Up & Turn on Thermostat\\
			\hline
			Interpreter & Air Condition\\
			\hline
			Open Door & Close Door\\
			\hline
		\end{tabular}
	\end{center}
	\caption{Selected Commands for data collection and experiment}
	\label{table:asl_signs}
\end{table}

\subsection{Preprocessing}
Kinect skeleton sequences are captured in camera coordinate system \ie, camera position is the origin. To make sequences invariant to different camera positions and body-shapes we convert sequences into body coordinate system. We pick a joint location and make it origin, then we subtract $(x, y, z)$ locations of all other joints from it. Finally, different collected sample has different frame length. The video file with $32$ frame rate has negligible difference between two consecutive frames. Hence, we sampled a fix number of frames from each sample. We used this fixed number \textit{T} to be $20$. 

\begin{figure*}
	\begin{center}
		%		\fbox{\rule{0pt}{2in} \rule{.9\linewidth}{0pt}}
		\includegraphics[width=.8\linewidth]{ac_person1_faceoff}
	\end{center}
	\caption{Example of a sign (Air Condition) performed by a person. Top row shows six frames from RGB videos increasing in time from left to right. Bottom row shows exactly same frames but with skeleton data visualization on it. Face of subject intentionally made unrecognizable}
	\label{fig:ac_person1}
\end{figure*}

\section{Sequential deep learning for ASL}
In this section, we are going to describe several sequential deep learning models and show how we can use those machine learning models for ASL recognition. Like most of the machine learning models, artificial neural networks (ANN) learn a mapping function from an input domain to an output. Usually, inputs to an ANN is a $1$D or multidimensional vector and output is a $1$D vector. The input data goes through a non-linear transformation along the network. Using the output vector, one can do some prediction or regression tasks. The parameters of the network are learned by feeding training data to it continuously. On each iteration, first the network predicts labels for training data, then calculates loss using actual labels and finally updates the parameters based on the calculated loss. This iteration process is repeated for a predefined number of times or until a desirable accuracy is acquired on test data. While updating parameters, it starts from the output layer and move backward to the input layer. As it progresses backward, it propagates loss and calculates gradients of errors with respect to the parameters of each layer in a systematic way. Formally this is known as back propagation algorithm and it is the heart of any neural network based machine learning model. 
Speaking of ANN, we usually imagine a feed forward neural network where it has an input layer, an output layer and one or more hidden layers. Parameters in different layers are different from each other. This type of network has shown tremendous performance accuracy in capturing non linear distinction among different classes in a dataset. Specifically, Convolutional neural network (CNN) which is one variant of feed forward neural network has shown human level accuracy for the image classification tasks ~\cite{NIPS2012_4824}. Images can be thought of as $2$D matrices. Patterns a machine learning model needs to learn from an image extends through this height and width dimensions. Unlike image recognition, some practical problems have also dependency over time \ie, information at certain point depends on information from a previous point or a future point or both. One very common example of such data is natural language. Say for example, we want to predict the next word in the blank position of the sentence ``Her mother is a school teacher. $\rule{1cm}{0.15mm}$ teaches in ABC high school''. We can easily see that the blank position is a pronoun which refers to the word \emph{mother} in the previous sentence. But if the word were \emph{father} instead of the word \emph{mother} then the pronoun at the ``$\rule{1cm}{0.15mm}$'' position would be changed too. This example shows an temporal dependency in data. Although feed forward neural networks have shown tremendous accuracy in finding spatial pattern in data, it can not efficiently capture such kind of temporal dependencies. One of main reasons behind this weakness is that it sees the whole input data at once and then it keeps transforming the data through its hidden layers until reaches the final layer. In other words it has to capture the patterns as a whole from the input data. We can solve this problem if there is a mechanism which first allows us to input a portion of data in an order and then models it. Then it takes as input the second portion of the ordered data, fuses it with modeled output from the first step and then models the fused data. Keep doing this until the final portion of our ordered data will allow us to capture temporal dependencies in our data. To serve this purpose Recurrent neural network (RNN) was introduced ~\cite{DBLP:journals/corr/Lipton15}. It is one of the variations of ANN which is suitable for capturing temporal dynamics of data. Figure ~\ref{fig:rnn_network} shows a high level view of such network. 
\begin{equation}
\label{eq:rnn_eq}
\begin{aligned}
	g(x_t, h_{t-1}) & = Ux_t + Vh_{t-1} \\ f & = \phi(g)
\end{aligned}
\end{equation}
Eq~\ref{eq:rnn_eq} shows the equations of an usual RNN. Here $x_t$ is the current input and $h_{t-1}$ is the previous RNN state. $U$ and $V$ are the network parameters associated with $x_t$ and $h_{t-1}$ respectively. $\phi$ is a nonlinear function which we can choose from several alternatives. Some common choices are $sigmoid$, $ReLu$, $tanh$ etc.
Two of the most important differences between RNN and feed forward neural networks are time ordered input and network parameter sharing. We can see from Figure~\ref{fig:rnn_network} that input is fed to the network in different time steps, not at once like feed forward case. However the parameters the network learns throughout different time steps are same. At any time step the network process current input along with whatever previously seen and processed data. This helps to capture the temporal dynamics in data. But the basic RNN has problem dealing with long term dependencies in data ~\cite{DBLP:journals/corr/abs-1211-5063}. This happens because the basic RNN has no control over what it should learn and what should forget. It just keeps updating its memory as it goes which is not desirable in most practical cases. Sometimes a memory from the long past could be useful for current prediction than a memory from the recent past. Another problem which makes training an RNN problematic is known as vanishing gradient problem ~\cite{DBLP:journals/corr/abs-1211-5063}. Since network parameters of an RNN are shared over time, at any time step the error derivative not only depends on the current input of network but also on the previous state of the network. So there is a multiplicative term of error derivatives back to time step $0$ to calculate error derivative of current time. This is a modification of back propagation algorithm and known as back propagation through time (BPTT) ~\cite{58337}. If individual gradients are close to zero this multiplicative term would become very close to zero and eventually gradients disappear. The opposite can also happen, which is called exploding gradient problem, however this is more obvious and easy to detect. The vanishing gradient problem is not easy to detect and serves as a silent killer to the network. Several approaches have been adopted to deal with vanishing gradient problem. Some of them are to careful initialization of network parameters, early stopping etc ~\cite{DBLP:journals/corr/abs-1211-5063}. But most effective solution was to modify the RNN architecture so that it has its own memory and using that memory it can control what to remember and what to forget. This architecture is called long short term memory (LSTM) network ~\cite{Hochreiter:1997:LSM:1246443.1246450}. Another similar architecture called gated recurrent unit (GRU) has been also proposed and has shown robustness against vanishing/exploding gradient problem~\cite{DBLP:journals/corr/ChoMGBSB14}.

\begin{figure}
	\begin{center}
		%\fbox{\rule{0pt}{2in} \rule{0.9\linewidth}{0pt}}
		\includegraphics[width=\linewidth]{rnn_network}
	\end{center}
	\caption{An RNN networks unrolled over T time steps. At each time step the network has one raw input and one processed input from previous time step. Combining these two it produces embedding for current time step}
	\label{fig:rnn_network}
\end{figure}

\subsection{Long short term memory} \label{lstm_sec}
As we see from eq  ~\ref{eq:rnn_eq}, at each time step, traditional RNN can be seen as a nonlinear function of the current input and the previous step. While vanilla RNN is a direct transformation from the previous state and the current input, LSTM takes a different approach. It maintains an internal memory and it has mechanism to update and use that memory. The mechanism consists of four separate neural networks also called gates. Figure~\ref{fig:lstm_cell} shows a cell of an LSTM network. 
%\begin{figure}[h]
%	\begin{center}
%		%\fbox{\rule{0pt}{2in} \rule{0.9\linewidth}{0pt}}
%		\includegraphics[width=\linewidth]{lstm_cell}
%	\end{center}
%	\caption{An LSTM cell}
%	\label{fig:lstm_cell}
%\end{figure}

Forget, input, update and output gate are four different gates represented by four circles and symbolized as $f_t$, $i_t$, $\tilde{C_t}$ and $o_t$ respectively. $\oplus$ and $\otimes$ represents element wise addition and multiplication respectively. Two vertical bars inside the rounded rectangle means concatenation operation on its inputs. 
\begin{equation}
\label{eq:lstm_eq}
	\begin{aligned}
		h_t &= o_t \otimes tanh(C_t) \\
		o_t &= \sigma(W_o * concat(h_{t-1}, x_t)) \\
		C_t &= (f_t \otimes C_{t-1}) \oplus (i_t \otimes \tilde{C}_t)  \\
		i_t &= \sigma(W_i * concat(h_{t-1}, x_t)) \\
		\tilde{C}_t &= tanh(W_{\tilde{C}} * concat(h_{t-1}, x_t))
	\end{aligned}
\end{equation}
Figure~\ref{fig:lstm_cell} shows input at the current time step $x_t$ and the previous state $h_{t-1}$ enter into the cell and get concatenated. Then the forget gate processes it to remove unnecessary information and outputs $f_t$ which gets multiplied with the previously stored memory $C_t$ and produces a refined memory for the current time. Meanwhile, the input gate and the update gate process the concatenated input and convert it into a candidate memory for the current time step. The refined memory from the previous step and proposed candidate memory of the current step get added to produce the final memory for the current step. This addition could render the output out of scale. Because of that, a squashing function is followed which is a $hyperbolic tan$ in our case. Its job is to scale the elements of the output vector into a fix range. Finally $o_t$, the output from output gate gets multiplied with output from the ♣squashing function and produces the output for the current time step. 
\begin{figure}[h]
	\begin{center}
		\begin{subfigure}{.45\textwidth}
			\includegraphics[width=\linewidth, height=.2\textheight]{lstm_cell2}
			\caption{}
			\label{fig:lstm_cell} 
		\end{subfigure}
		
		\begin{subfigure}{.45\textwidth}
			\includegraphics[width=\linewidth, height=.2\textheight]{gru_cell2}
			\caption{}
			\label{fig:gru_cell} 
		\end{subfigure}
	\end{center}
	\caption{LSTM and GRU cell}
	\label{fig:lstm_gru_cells}
\end{figure}
\begin{figure*}[h]
	\begin{center}
		%\fbox{\rule{0pt}{2in} \rule{0.9\linewidth}{0pt}}
		\includegraphics[width=.8\linewidth]{rnn_impl}
	\end{center}
	%	\caption{Implementaion of RNN models x, y, z}
	\caption{Implementation of RNN models to recognize ASL from skeletal data. Input at each time step is a concatenation of $3D$ coordinates of $12$ joints. $P_i$ is the $(x, y, z)$ of $i^{th}$ joint. Subscript of each input vector represents time step. We used $20$ time steps for each sample data in our experiments}
	\label{fig:rnn_impl}
\end{figure*}
\subsection{Gated recurrent unit}
A gated recurrent unit (GRU) is another variation of RNN for modeling temporal dynamics of data. It works in the same way as LSTM, but has a simpler construction. Like LSTM it has the mechanism to control flow of information over time, but it lacks the separate memory cell. 
Figure \ref{fig:gru_cell} shows an example of GRU cell. Symbols represent similar things as described in section ~\ref{lstm_sec}. 
%\begin{figure}[h]
%	\begin{center}
%		%\fbox{\rule{0pt}{2in} \rule{0.9\linewidth}{0pt}}
%		\includegraphics[width=\linewidth]{gru_cell}
%	\end{center}
%	\caption{A GRU cell}
%	\label{fig:gru_cell}
%\end{figure}
To better understand GRU cell, first we have to look at the output of it. The output is formed by a weighted linear sum of two components $\tilde{h}_t$ and $h_{t-1}$ as seen from the Eq.~\ref{eq:gru_eq}. Here $concat(x,y)$ means concatenation of input vectors $x$ and $y$. Element wise multiplication and addition are represented by $\otimes$ and $\oplus$ respectively.
\begin{equation}
	\label{eq:gru_eq}
	\begin{aligned}
		h_t & = (1-z_{t})h_{t-1} \oplus z_{t}\tilde{h}_t \\
		z_{t} &= \sigma(W_z*concat(h_{t-1}, x_t)) \\
		\tilde{h}_t &= tanh(W_{\tilde{h}}*concat(r_t \otimes h_{t-1}, x_t)) \\
		r_t &= \sigma(W_r*concat(h_{t-1}, x_t)) \\
	\end{aligned}
\end{equation}
The term $\tilde{h}_t$ represents newly proposed candidate embedding and $h_{t-1}$ is simply the previous state. Candidate embedding $\tilde{h}_t$ is a function of the current input and purified previous state. This purification is achieved by the reset gate $r_{t}$ whose job is to decide what portion of previous state should contribute to candidate embedding $\tilde{h}_t$. Reset gate $r_{t}$ itself is a separate neural network and its output close to $0$ means network takes a small portion of previous state to form a current candidate state embedding. The update gate $z_{t}$ decides what fractions of the candidate state and the previous state comprise the new state. It does so by taking a weighted some of the candidate state and the previous state. The weights are $z_{t}$ and $1-z_{t}$. This whole new state is outputted where in case of LSTM, the output of the new state is controlled by an output gate as shown in Eq. ~\ref{eq:lstm_eq}.  

\subsection{Our approach}
ASL signs are usually performed using two hands, head movements and in some cases facial expressions. Motion dynamics of two hands are primarily responsible for creating different cues for various signs. Each joint in two hands and face area follows a specific sequential pattern for a particular gesture sign. For example, from Figure~\ref{fig:ac_person1} we see the person is performing gesture for \textit{air condition}. We can see that, for this sign first, one has to lift and show palm area of the left hand and then he has to lift both hands and make a back and forth motion with all fingers. This is basically a sequence of joints in both hands. Given the efficacy of variants of RNN in encoding sequential pattern in data, we want to see how those models perform on sequence data like ASL sign. To achieve this goal, we implemented LSTM and GRU model for ASL recognition problem. We notice that, the joints located above the waist are mostly useful in case of sign language. Hence, we take only $12$ joints into consideration in our experiment. These joints come from both hands, head, neck and spine area. Figure~\ref{fig:rnn_impl} shows our implementation. For each time step, first, we concatenate $3D$ coordinates of $12$ joints and then feed to our RNN network. Finally, we take the final state of the network and send it to a softmax layer and get prediction probabilities over our classes.
\section{Experiments}
In this section, we are going to describe all of our experimental designs. First we are going to explain how we evaluated model performances. We are also going to describe a calibration mechanism we proposed to find a good seed percentage of train data for test person. Then we are going to present implementation details and results we got from experiments. 
\begin{figure*}[h]
	\begin{center}
		%\fbox{\rule{0pt}{2in} \rule{0.9\linewidth}{0pt}}
		\includegraphics[width=.8\linewidth]{sk_data_viz}
	\end{center}
	%	\caption{Implementaion of RNN models x, y, z}
	\caption{Skeleton visualization of classes in the dataset -- a. Air condition, b. Wake up, c. Interpreter, d. Door open, e. Door close, f. Thermostat on}
	\label{fig:sk_dat_viz}
\end{figure*}

\subsection{Evaluation criteria}
Our experiment on the dataset is two fold. First, we did experimentation with two deep learning sequence models, namely GRU and LSTM. Also, we proposed a calibration mechanism which essentially infers what fraction of data from an unseen test subject is needed to boost performance of our models. This calibration experiment is done in cross subject manner \eg we train our models on some subjects and test on a different subject. This makes a lot of sense in real life situation where a trained model has to perform well on data it has not seen yet. 

\begin{table}[h]
	\begin{center}
		\begin{tabular}{|m{1cm}|m{5cm}|}
			\hline
			Terms & Description\\
			\hline\hline
			$D_{train}$ & Training data, whole data from the subjects other than the test subject \\
			\hline
			$D_{test}$ & Test data, $50\%$ data from test subject used for testing models\\
			\hline
			$D_{seed}$ & Seed data, $50\%$ data from test subject used for boosting up model's performance on $D_{test}$\\
			\hline
		\end{tabular}
	\end{center}
	\caption{Different terminologies regarding data split for one test subject}
	\label{table:terminology_eval}
\end{table}
Table ~\ref{table:terminology_eval} represents some terminologies helpful to understand calibration mechanism described next.

We have data on four different subjects. First, we choose a test subject and take out $50\%$ of data from this subject as test data say $D_{test}$. We use other $50\%$ of data as seed data say, $D_{seed}$ to train model along with the data from other three subjects say, $D_{train}$. We first train the model only with $D_{train}$ and test on $D_{test}$. Then we train our model with $D_{train}$ and $10\%$ from $D_{seed}$ and again test on $D_{test}$. Then again we train the model with $D_{train}$ and this time $20\%$ of $D_{seed}$ and test on $D_{test}$. As we increase the seed data in our training data, we should expect better test accuracy. We keep doing this until we use up to full $D_{seed}$ for training with $D_{train}$ and record different testing results on $D_{test}$. We repeat the whole procedure by taking each subject as a test subject one at a time. We want to see here what fraction of seed data is required during training to get a good performance from our model. Figure ~\ref{fig:train_test_dat} shows data splitting strategy for better understanding.

\begin{figure}[h]
	\begin{center}
		%\fbox{\rule{0pt}{2in} \rule{0.9\linewidth}{0pt}}
		\includegraphics[width=\linewidth]{train_test_dat}
	\end{center}
	\caption{Train and test data split for a test person}
	\label{fig:train_test_dat}
\end{figure}
\subsection{Results and discussions}
We got an almost similar types of results from both of LSTM and GRU implementation. However, result for DTW implementation is little different. We found that $30\%$ to $50\%$ data from the test subject as seed is good enough to give a performance boost. For DTW, accuracy is relatively lower with $0\%$ seed. But, with small amount of seed ($10\%-20\%$), it reaches highest accuracy for itself. This behavior is desirable because, DTW compares distances of one sample of test set with every other sample in training set and with little introduction of data from same class and subject (seed) would always give minimum distance between test sample and seed. Table~\ref{table:result_table} shows the results for each combination of a test person, a \% of seed data used in training for that person and a network type.
\begin{table}[h]
	\begin{center}
		\begin{tabular}{|m{1em} | m{3em} | m{1.5em} | m{1.5em} | m{1.5em} |m{1.5em} |m{1.5em} |m{1.5em} |}
			\hline
			%			 & \multicolumn{6}{c}{\% of training data} \vline\\
			%			\hline
			& \% seed & $0\%$ & $10\%$ & $20\%$ & $30\%$ & $40\%$ & $50\%$\\
			\hline
			\multirow{3}{3.5em}{$P_1$}
			& DTW & $0.66$ & $0.89$ & $0.95$ & $0.96$ & $0.96$ & $0.96$\\
			\cline{2-8}
			& LSTM & $0.86$ & $\textbf{1.00}$ & $\textbf{1.00}$ & $\textbf{1.00}$ & $0.99$ & $\textbf{1.00}$\\
			\cline{2-8}
			& GRU & $0.78$ & $0.98$ & $0.95$ & $\textbf{1.00}$ & $\textbf{0.97}$ & $\textbf{0.98}$\\
			\hline
			\multirow{3}{3.5em}{$P_2$}
			& DTW & $0.85$ & $0.89$ & $0.90$ & $0.91$ & $0.94$ & $0.95$\\
			\cline{2-8}
			& LSTM & $0.90$ & $0.93$ & $0.96$ & $0.93$ & $\textbf{0.98}$ & $0.97$\\
			\cline{2-8}
			& GRU & $0.93$ & $0.93$ & $0.93$ & $0.93$ & $\textbf{0.98}$ & $\textbf{0.98}$\\
			\hline
			\multirow{3}{3.5em}{$P_3$}
			& DTW & $0.50$ & $0.54$ & $0.82$ & $0.94$ & $0.96$ & $0.96$\\
			\cline{2-8}
			& LSTM & $0.72$ & $0.65$ & $0.72$ & $0.82$ & $0.82$ & $\textbf{0.86}$\\
			\cline{2-8}
			& GRU & $0.62$ & $0.72$ & $0.75$ & $0.86$ & $0.86$ & $\textbf{0.90}$\\
			\hline
			\multirow{3}{3.5em}{$P_4$}
			& DTW & $0.73$ & $0.91$ & $0.91$ & $0.91$ & $0.91$ & $0.91$\\
			\cline{2-8}
			& LSTM & $0.89$ & $0.93$ & $0.93$ & $0.94$ & $0.93$ & $\textbf{0.96}$\\
			\cline{2-8}
			& GRU & $0.91$ & $0.94$ & $0.94$ & $\textbf{0.96}$ & $0.95$ & $0.95$\\
			\hline
			\multirow{3}{3.5em}{$\bar{P}$}
			& DTW & $0.69$ & $0.81$ & $0.90$ & $0.93$ & $0.94$ & $0.96$\\
			\cline{2-8}
			& LSTM & $0.84$ & $0.88$ & $0.90$ & $0.92$ & $0.93$ & $\textbf{0.95}$\\
			\cline{2-8}
			& GRU & $0.81$ & $0.89$ & $0.93$ & $\textbf{0.94}$ & $0.94$ & $0.95$\\
			\hline
		\end{tabular}
	\end{center}
	\caption{Test accuracies for each combination of test person, network type and \% of data from test subject used in training. $P_i$ denotes the $i^{th}$ person and $\bar{P}$ denotes the average accuracy of all subjects}
	\label{table:result_table}
\end{table}

\begin{figure}
	\begin{center}
		\begin{subfigure}{.45\textwidth}
			\includegraphics[width=\linewidth, height=.25\textheight]{cm_zero_percent_cropped}
			\caption{}
			\label{fig:cm_zero} 
		\end{subfigure}
		
		\begin{subfigure}{.45\textwidth}
			\includegraphics[width=\linewidth, height=.25\textheight]{cm_thirty_percent_cropped}
			\caption{}
			\label{fig:cm_thirty} 
		\end{subfigure}
	\end{center}
	\caption{Confusion matrix for $0\%$ (a) and $30\%$ (b) seed data.}
	\label{fig:conf_mat}
\end{figure}


\begin{table}[h]
	\begin{center}
		\begin{tabular}{|c|c|}
			\hline
			Model type & Test accuracy\\
			\hline\hline
			LSTM & $\textbf{0.94}$\\
			\hline
			GRU & $\textbf{0.91}$\\
			\hline
		\end{tabular}
	\end{center}
	\caption{Results from $70\%$--$30\%$ train-test split}
	\label{table:result_conv}
\end{table}
%\begin{figure*}
%	\begin{center}
%		\begin{subfigure}{.8\textwidth}
%			\includegraphics[width=\linewidth, height=.3\textheight]{result_lstm}
%			\caption{}
%			\label{fig:result_lstm} 
%		\end{subfigure}
%		
%		\begin{subfigure}{.8\textwidth}
%			\includegraphics[width=\linewidth, height=.3\textheight]{result_gru}
%			\caption{}
%			\label{fig:result_gru} 
%		\end{subfigure}
%	\end{center}
%	\caption{Test accuracy. X-axis represents percentage of data from the test subject used in training and Y-axis is the test accuracy on $50\%$ hold out data for test subject. (a) Test accuracy for LSTM network. Each plot represents accuracy on test data for each subject in our dataset. (b) Test accuracy for GRU network in same way as (a).}
%	\label{fig:result_comp}
%\end{figure*}
We observe from the result that the highest accuracy is obtained using $30\%$ or more seed data. This suggests that even if we have a small amount of seed data for a test persons, our system can boost up to a higher accuracy. In a practical scenario, system will be able to recognize new subjects correctly within a short period of subject's interaction with the system. 
%Figure~\ref{fig:result_comp} shows plots of test accuracy vs percentage of data used in training. It shows the LSTM results at the top and the GRU results at the bottom. Plots are also suggesting that $30\%$ or more of seed data is good enough for a performance increase to the model. We also analyzed confusion matrix to see what classes confuses each other and found that increasing percentage of seed data in training decreases confusion. 
Figure~\ref{fig:cm_zero} shows confusion matrix when no seed data is used in training. We can see that sign ``Close Door'' confuses with signs ``Thermostat On'' and ``Open Door''. However, in Figure~\ref{fig:cm_thirty} we observe that, with only increase of seed data to $30\%$ in training confusion reduces significantly. We can see from skeleton visualization in figure ~\ref{fig:sk_dat_viz} that, ``Close Door'', ``Thermostat On'' and ``Open Door'' are somewhat similar in terms of joint movements. That's why the confusion occurs. But, with small amount of seed data, models learn to capture minute details and gain higher accuracy.

We also tested our model in conventional train test split manner. In that case, we took $30\%$ of whole data as test data. Table~\ref{table:result_conv} shows accuracy for both type of models after performing k-fold cross validation.

 
\subsection{Implementation and parameter selection}
We used same value for hyper-parameters such as learning rate, state size of RNNs, batch size etc. for both compared models. We did a grid search for finding best hyper-parameters. Our experiments found the best learning rate as $5e^{-4}$ and the state size for RNNs as $36$. For each joint state size is $3$ which gives us the whole state size of $36$ for $12$ joints. We used Adam Optimizer for optimizing our networks~\cite{DBLP:journals/corr/KingmaB14}. We used this optimizer because it has shown superior performance in training different types of network. It also adapts learning rate and decay rate as training continues. We used mini-batch size of $1$ which means the network updates it's parameter after seeing one training example. We keep training our network until a good accuracy is achieved. However, to battle over-fitting of the network, we followed early stopping strategy. If there is no increase in the testing accuracy for $40$ epochs, we decided to stop training. We implemented our models using TensorFlow deep learning library version $1.7$ \footnote{https://www.tensorflow.org/}. These experiments were run on ARGO, a research computing cluster provided by the Office of Research Computing at George Mason University, VA \footnote{URL:http://orc.gmu.edu}. 

\section{Conclusion}
Here goes conclusion


%All text must be in a two-column format. The total allowable width of the
%text area is $6\frac78$ inches (17.5 cm) wide by $8\frac78$ inches (22.54
%cm) high. Columns are to be $3\frac14$ inches (8.25 cm) wide, with a
%$\frac{5}{16}$ inch (0.8 cm) space between them. The main title (on the
%first page) should begin 1.0 inch (2.54 cm) from the top edge of the
%page. The second and following pages should begin 1.0 inch (2.54 cm) from
%the top edge. On all pages, the bottom margin should be 1-1/8 inches (2.86
%cm) from the bottom edge of the page for $8.5 \times 11$-inch paper; for A4
%paper, approximately 1-5/8 inches (4.13 cm) from the bottom edge of the
%page.

%-------------------------------------------------------------------------

{\small
\bibliographystyle{ieee}
\bibliography{egbib}
}


\end{document}
